\documentclass{article}%
\usepackage[T1]{fontenc}%
\usepackage[utf8]{inputenc}%
\usepackage{lmodern}%
\usepackage{textcomp}%
\usepackage{lastpage}%
%
\usepackage{amsmath, amssymb, geometry}%
\geometry{a4paper, margin=1in}%
%
\begin{document}%
\normalsize%
\section*{Problem}%
Let \]ABC\] be a triangle inscribed in circle \]\omega\]. Let the tangents to \]\omega\] at \]B\] and \]C\] intersect at point \]D\], and let \]\overline{AD}\] intersect \]\omega\] at \]P\]. If \]AB=5\], \]BC=9\], and \]AC=10\], \]AP\] can be written as the form \]\frac{m}{n}\], where \]m\] and \]n\] are relatively prime integers. Find \]m + n\].%
\section*{Solution}%
From the tangency condition we have \]\let\angle BCD = \let\angle CBD = \let\angle A\]. With LoC we have \]\cos(A) = \frac{25+100-81}{2*5*10} = \frac{11}{25}\] and \]\cos(B) = \frac{81+25-100}{2*9*5} = \frac{1}{15}\]. Then, \]CD = \frac{\frac{9}{2}}{\cos(A)} = \frac{225}{22}\]. Using LoC we can find \]AD\]: \]AD^2 = AC^2 + CD^2 - 2(AC)(CD)\cos(A+C) = 10^2+(\frac{225}{22})^2 + 2(10)\frac{225}{22}\cos(B) = 100 + \frac{225^2}{22^2} + 2(10)\frac{225}{22}*\frac{1}{15} = \frac{5^4*13^2}{484}\]. Thus, \]AD = \frac{5^2*13}{22}\]. By Power of a Point, \]DP*AD = CD^2\] so \]DP*\frac{5^2*13}{22} = (\frac{225}{22})^2\] which gives \]DP = \frac{5^2*9^2}{13*22}\]. Finally, we have \]AP = AD - DP = \frac{5^2*13}{22} - \frac{5^2*9^2}{13*22} = \frac{100}{13} \rightarrow {113}\].
~angie.
We know \]AP\] is the symmedian, which implies \]\triangle{ABP}\sim \triangle{AMC}\] where \]M\] is the midpoint of \]BC\]. By Appolonius theorem, \]AM=\frac{13}{2}\]. Thus, we have \]\frac{AP}{AC}=\frac{AB}{AM}, AP=\frac{100}{13}\Rightarrow {113}\]
~Bluesoul
Extend sides \]\overline{AB}\] and \]\overline{AC}\] to points \]E\] and \]F\], respectively, such that \]B\] and \]C\] are the feet of the altitudes in \]\triangle AEF\]. Denote the feet of the altitude from \]A\] to \]\overline{EF}\] as \]X\], and let \]H\] denote the orthocenter of \]\triangle AEF\]. Call \]M\] the midpoint of segment \]\overline{EF}\]. By the Three Tangents Lemma, we have that \]MB\] and \]MC\] are both tangents to \](ABC)\] \]\Rightarrow\] \]M = D\], and since \]M\] is the midpoint of \]\overline{EF}\], \]MF = MB\]. Additionally, by angle chasing, we get that: 
\[\angle ABC \cong \angle AHC \cong \angle EHX\]
Also, 
\[\angle EHX = 90 ^\circ - \angle HEF = 90 ^\circ - (90 ^\circ - \angle AFE) = \angle AFE\] 
Furthermore, 
\[AB = AF \cdot \cos(A)\]
From this, we see that \]\triangle ABC \sim \triangle AFE\] with a scale factor of \]\cos(A)\]. By the Law of Cosines, 
\[\cos(A) = \frac{10^2 + 5^2 - 9^2}{2 \cdot 10 \cdot 5} = \frac{11}{25}\] 
Thus, we can find that the side lengths of \]\triangle AEF\] are \]\frac{250}{11}, \frac{125}{11}, \frac{225}{11}\]. Then, by Stewart's theorem, \]AM = \frac{13 \cdot 25}{22}\]. By Power of a Point, 
\[\overline{MB} \cdot \overline{MB} = \overline{MA} \cdot \overline{MP}\]
\[\frac{225}{22} \cdot \frac{225}{22} = \overline{MP} \cdot \frac{13 \cdot 25}{22} \Rightarrow \overline{MP} = \frac{225 \cdot 9}{22 \cdot 13}\]
Thus, 
\[AP = AM - MP = \frac{13 \cdot 25}{22} - \frac{225 \cdot 9}{22 \cdot 13} = \frac{100}{13}\]
Therefore, the answer is \]{113}\].
~mathwiz_1207
Connect lines \]\overline{PB}\] and \]\overline{PC}\]. From the angle by tanget formula, we have \]\angle PBD = \angle DAB\]. Therefore by AA similarity, \]\triangle PBD \sim \triangle BAD\]. Let \]\overline{BP} = x\]. Using ratios, we have \[\frac{x}{5}=\frac{BD}{AD}.\] Similarly, using angle by tangent, we have \]\angle PCD = \angle DAC\], and by AA similarity, \]\triangle CPD \sim \triangle ACD\]. By ratios, we have \[\frac{PC}{10}=\frac{CD}{AD}.\] However, because \]\overline{BD}=\overline{CD}\], we have \[\frac{x}{5}=\frac{PC}{10},\] so \]\overline{PC}=2x.\] Now using Law of Cosines on \]\angle BAC\] in triangle \]\triangle ABC\], we have \[9^2=5^2+10^2-100\cos(\angle BAC).\] Solving, we find \]\cos(\angle BAC)=\frac{11}{25}\]. Now we can solve for \]x\]. Using Law of Cosines on \]\triangle BPC,\] we have 
\begin{align*}
81&=x^2+4x^2-4x^2\cos(180-\angle BAC) \\ 
&= 5x^2+4x^2\cos(BAC). \\
\end{align*}
Solving, we get \]x=\frac{45}{13}.\] Now we have a system of equations using Law of Cosines on \]\triangle BPA\] and \]\triangle CPA\], \[AP^2=5^2+\left(\frac{45}{13}\right)^2 -(10) \left(\frac{45}{13} \right)\cos(ABP)\]
\[AP^2=10^2+4 \left(\frac{45}{13} \right)^2 + (40) \left(\frac{45}{13} \right)\cos(ABP).\]
Solving, we find \]\overline{AP}=\frac{100}{13}\], so our desired answer is \]100+13={113}\].
~evanhliu2009
Following from the law of cosines, we can easily get \]\cos A = \frac{11}{25}\], \]\cos B = \frac{1}{15}\], \]\cos C = \frac{13}{15}\].
Hence, \]\sin A = \frac{6 \sqrt{14}}{25}\], \]\cos 2C = \frac{113}{225}\], \]\sin 2C = \frac{52 \sqrt{14}}{225}\].
Thus, \]\cos \left( A + 2C \right) = - \frac{5}{9}\].
Denote by \]R\] the circumradius of \]\triangle ABC\].
In \]\triangle ABC\], following from the law of sines, we have \]R = \frac{BC}{2 \sin A} = \frac{75}{4 \sqrt{14}}\].
Because \]BD\] and \]CD\] are tangents to the circumcircle \]ABC\], \]\triangle OBD \cong \triangle OCD\] and \]\angle OBD = 90^\circ\].
Thus, \]OD = \frac{OB}{\cos \angle BOD} = \frac{R}{\cos A}\].
In \]\triangle AOD\], we have \]OA = R\] and \]\angle AOD = \angle BOD + \angle AOB = A + 2C\].
Thus, following from the law of cosines, we have
\begin{align*}
AD & = \sqrt{OA^2 + OD^2 - 2 OA \cdot OD \cos \angle AOD} \\
& = \frac{26 \sqrt{14}}{33} R.
\end{align*}

Following from the law of cosines,
\begin{align*}
\cos \angle OAD & = \frac{AD^2 + OA^2 - OD^2}{2 AD \cdot OA} \\
& = \frac{8 \sqrt{14}}{39} .
\end{align*}

Therefore,
\begin{align*}
AP & = 2 OA \cos \angle OAD \\
& = \frac{100}{13} .
\end{align*}

Therefore, the answer is \]100 + 13 = {\textbf{(113) }}\].
~Steven Chen (Professor Chen Education Palace, www.professorchenedu.com)%
\section*{Answer}%
113%
\end{document}